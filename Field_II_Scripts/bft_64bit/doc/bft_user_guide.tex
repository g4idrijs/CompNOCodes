%\documentclass[11pt]{report}
\documentclass{manual}

%%tth:\begin{html}<title>Beamformation Toolbox</title>\end{html}

%  version 1.0, 29/2-1999, SN

%  Set TeX to use only Post Script fonts for both the
%  text and the equations

\usepackage{times}
\usepackage{mathptm}
\usepackage{longtable}
\usepackage[pdftex]{graphics}   %  For including links in a pdf document.
\usepackage{geometry}
\usepackage[colorlinks, linkcolor = BrickRed]{hyperref}
\usepackage{epsfig}

%  Set the text height and width

\textwidth 16cm
%\textheight 24cm
%\oddsidemargin 0cm
%\evensidemargin 0cm
%\topmargin -1cm

%  Set the indentation for each new paragraph

\setlength{\parindent}{0ex}
\setlength{\parskip}{2ex} 

%  Define a macro for making headers for each procedure description

\def\thesubsection{\huge }
\def\boldindex#1{\textbf{\hyperpage{#1}}}

\newcommand{\funlnk}[1]
{
  \label{#1}
  \hypertarget{#1}{}
}

\newcommand{\headline}[1]
{
% \clearpage
  \subsection[#1]{}
  \setlength{\unitlength}{1cm}
  \begin{center}
  \begin{picture}(15.9,1.7)
    \thicklines
    \put(0,0){\framebox(15.9,1.3)}
    \put(0,0){\makebox(15.9,1.3)[l]{\Large\em \hspace{0.2 cm} BFT user's guide}}
    \put(0,0){\makebox(15.9,1.3)[r]{\Large\bf #1 \hspace{0.2cm}}}
   \end{picture}
   \end{center}
  % \vspace{0.2cm}
}

%  Create some useful macros
%  Insert a figure, where the size is determined in the text

\newcommand{\insertfigure}[2]
{
  \setlength{\epsfxsize}{#2} 
  \begin{center}
    \mbox{\epsfbox{#1}}
  \end{center}
  \vspace*{-0.5cm}
%  \begin{center}
%    \epsfig{file=#1, height=#2}
%  \end{center}
}
\title{Users' guide for the 
Beamformation Toolbox}
\author{Svetoslav Ivanov Nikolov}
\authoraddress{
Department of Information Technology, Build. 344, \\
Technical University of Denmark \\
DK-2800 Lyngby, Denmark
}

\date{Aug 31 2000}

\release{1.3}


\begin{document}           % End of preamble and beginning of text.
\definecolor{BrickRed}{rgb}{0.5 , 0 , .3}

%\input epsf
 
\maketitle                 % Produces the title.

%\pdfbookmark{Contents}{content}
\tableofcontents
%%%tth:\tableofcontents
%%tth:\begin{html}<hr>\end{html}
%%tth:Chapter 1.\\ ~~~\hyperlink{chap_intro}{Introduction} \\
%%tth:Chapter 2.\\ ~~~\hyperlink{chap_func_desc}{Description of Matlab functions} \\
%%tth:Chapter 3.\\ ~~~\hyperlink{chap_func_desc}{Description of Matlab functions} \\
%%tth:Chapter 4.\\ ~~~\hyperlink{chap_examples}{Examples}\\
%%tth:\begin{html}<hr>\end{html}

%%tth:\begin{html}<hr>\end{html}
\chapter{Introduction}
\label{chap_intro}
%%tth:\begin{html}<hr>\end{html}

 This is the user guide to the Beamformation Toolbox. This toolbox is
available only for Matlab under Linux, and is intended for processing 
raw RF data recorded by an ultrasound system. Typical applications include
synthetic aperture focusing and beamformation of data recorded by 
experimental systems such as XTRA or the \emph{Experimental Ultrasound 
System for Real Time Synthetic Aperture Focusing}. 
All the functions have calling conventions like 
{\bf \href{http://www.it.dtu.dk/~jaj/field}{Field-II}}.

The default behaviour of the toolbox is to beamform only one line at a time, 
also like in Field II. In this case the user can omit the last of the 
arguments in all the functions, which usually is {\sl line\_no}.

However, there are some differences between this toolbox and {\bf \href{http://www.it.dtu.dk/~jaj/field}{Field-II}}. 
For the purposes of the synthetic aperture focusing it is necessary to 
beamform a whole low-resolution image at every emission. This can be 
achieved by setting the number of simultaneously beamformed lines using the 
command \hyperlink{bft_no_lines}{\tt bft\_no\_lines}. Then each of the lines
must be described before calling \hyperlink{bft_beamform}{\tt bft\_beamform}.

For the time-being there are some limitations :
\begin{itemize}
\item Two low-resolution images can not be added if the lines are specified 
       as dynamically focused.        
\end{itemize}

 As it can be seen the library is fully functional when the focus lines 
are specified by a number of focal points\footnote{There is a speed difference 
when the apodization is set to \emph{ones} or if no apodization is set at all.},
and for the case of dynamic focusing, low resolution images cannot be summed.
 
In the future the toolbox will try to support many different types of strange and wacky 
algorithms, so - stay tuned. 


\vspace{1cm}
Svetoslav Ivanov Nikolov \\
February 29, 2000.


%%tth:\begin{html}<hr>\end{html}
\chapter{Description of Matlab Functions}
\label{chap_func_desc}
%%tth:\begin{html}<hr>\end{html}

\section{Functions list}
\begin{tabular}[t]{lp{14cm}}
 \hyperlink{bft_add_image}{\tt bft\_add\_image}      & Add a low resolution to hi resolution image.\\
 \hyperlink{bft_apodization}{\tt bft\_apodization}     & Create a  apodization time line. \\
 \hyperlink{bft_beamform}{\tt bft\_beamform}        & Beamform a number of scan-lines. \\
 \hyperlink{bft_center_focus}{\tt bft\_center\_focus}   & Set the center focus point for the focusing. \\
 \hyperlink{bft_dynamic_focus}{\tt bft\_dynamic\_focus}  & Set dynamic focusing for a line. \\
 \hyperlink{bft_end}{\tt bft\_end}             & Release all resources, allocated by the beamforming toolbox.\\
 \hyperlink{bft_focus}{\tt bft\_focus}           & Create a focus time line defined by focus points.\\
 \hyperlink{bft_focus_times}{\tt bft\_focus\_times}    & Create a focus time line defined by focus delays.\\
 \hyperlink{bft_free_xdc}{\tt bft\_free\_xdc}       & Free the memory allocated for a transducer definition.\\
 \hyperlink{bft_init}{\tt bft\_init}            & Initialize the BeamForming Toolbox.\\
 \hyperlink{bft_linear_array}{\tt bft\_linear\_array}   & Create a linear array.\\
 \hyperlink{bft_no_lines}{\tt bft\_no\_lines}       & Set the number of lines that will be beamformed in parallel.\\
 \hyperlink{bft_param}{\tt bft\_param}           & Set a paramater of the BeamForming Toolbox.\\
 \hyperlink{bft_sub_image}{\tt bft\_sub\_image}      & Subtract one low-res image from  high-res one.\\
 \hyperlink{bft_sum_apodization}{\tt bft\_sum\_apodization} & Create a summation apodization time line.\\
 \hyperlink{bft_sum_images}{\tt bft\_sum\_images}     & Sum 2 low resolution images in 1 high resolution.\\
 \hyperlink{bft_transducer}{\tt bft\_transducer}      & Create a new transducer definition.\\
\end{tabular}

\newpage
\section{Function description}
\headline{bft\_add\_image}
%%tth:\vspace{1cm}
%%tth:\begin{html}<hr>\end{html}
%%tth:\subsection*{bft\_add\_image}
%%tth:\begin{html}<hr>\end{html}
\funlnk{bft_add_image} 

Add a low resolution to high resolution image. 

\begin{tabular}[t]{lp{14cm}}  
USAGE: & {\verb+ [hi_res]=bft_add_image(hi_res,lo_res,element,start_time)+ }\\
INPUT: & \begin{tabular}[t]{lp{11cm}}
          {\sl hi\_res} & High resolution RF image. One column per scan line. \\
          {\sl lo\_res} & Low resolution RF image. One column per scan line. \\
          {\sl element} & Number of element, used to acquire the low resolution
                        image. \\
           {\sl time}   & Arrival time of the first sample of the RF lines. [sec]\\
         \end{tabular}\\
         
OUTPUT:& hi\_res - The high resolution image
\end{tabular}

%%%%%%%%%%%%%%%%%%%%%%%%%%%%%%%%%%%%%%%%%%%%%
\headline{bft\_appodization}
%%tth:\vspace{2cm}
%%tth:\begin{html}<hr>\end{html}
%%tth:\subsection*{bft\_appodization}
%%tth:\begin{html}<hr>\end{html}
\funlnk{bft_apodization}

Create an apodization time line.

\begin{tabular}[t]{lp{14cm}}  
 
  USAGE: & {\tt bft\_apodization(xdc, times, values, line\_no)} \\
 
  INPUT: & \begin{tabular}[t]{lp{11cm}}
          {\sl xdc} & Pointer to a transducer aperture \\
          {\sl times}  & Time after which the associated apodization is valid \\
          {\sl values} & Apodization values. Matrix with one row for each
                   time value and a number of columns equal to the 
                   number of physical elements in the aperture. \\
          {\sl line\_no} & Number of line. If skipped, {\sl line\_no} is assumed
                    to be equal to '1'.
          \end{tabular} \\
  OUTPUT: & None
\end{tabular}


%%%%%%%%%%%%%%%%%%%%%%%%%%%%%%%%%%%%%%%%
\headline{bft\_beamform}
%%tth:\vspace{2cm}
%%tth:\begin{html}<hr>\end{html}
%%tth:\subsection*{bft\_beamform}
%%tth:\begin{html}<hr>\end{html}
\funlnk{bft_beamform}

Beamform a number of scan-lines. The number of the simultaneously beamformed
scan-lines is set by \hyperlink{bft_no_lines}{\tt bft\_no\_lines}. 
If \hyperlink{bft_no_lines}{\tt bft\_no\_lines} is not called, only one scan
line will be beamformed.

\begin{tabular}[t]{lp{14cm}}  
 USAGE: & {\tt bf\_lines = bft\_beamform(time, rf\_data)} \\
 INPUT: & \begin{tabular}[t]{lp{11cm}}
          {\sl time}   & The time of the first sampled value \\
          {\sl rf\_data} & The recorded RF data. The number of columns 
                    is equal to the number of elements.
          \end{tabular}\\
 OUTPUT: & {\sl bf\_lines}  Matrix with the beamformed data. The number 
                    of rows of {\sl bf\_lines} is equal to the number 
                    of rows of {\sl rf\_data}. The number of columns 
                    is equal to the number of lines \\
 
\end{tabular}


%%%%%%%%%%%%%%%%%%%%%%%%%%%%%%%%%%%%%
\headline{bft\_center\_focus}
%%tth:\vspace{2cm}
%%tth:\begin{html}<hr>\end{html}
%%tth:\subsection*{bft\_center\_focus}
%%tth:\begin{html}<hr>\end{html}
\funlnk{bft_center_focus}

 Set the center focus point for the focusing.
  This point is used as a reference for calculating the
   focusing delay times and as a starting point for dynamic
    focusing.

\begin{tabular}[t]{lp{14cm}}  
  USAGE: & {\tt bft\_center\_focus(point, line\_no)} \\
 
  INPUT: & \begin{tabular}[t]{lp{11cm}}  
           {\sl point} & The center point [x,y,z]              [ m ]\\
           {\sl line\_no} & Number of line. If omitted in the parameter
                    list {\sl line\_no} is assumed equal to 1\\
          \end{tabular}\\
  OUTPUT: & None
\end{tabular}


%%%%%%%%%%%%%%%%%%%%%%%%%%%%%%%%%%
\headline{bft\_dynamic\_focus}
%%tth:\vspace{2cm}
%%tth:\begin{html}<hr>\end{html}
%%tth:\subsection*{bft\_dynamic\_focus}
%%tth:\begin{html}<hr>\end{html}
\funlnk{bft_dynamic_focus}

Set dynamic focusing for a line

\begin{tabular}[t]{lp{14cm}}  
 
  USAGE: & {\tt bft\_dynamic\_focus(xdc, dir\_xz, dir\_zy, line\_no)} \\
  
  INPUT: & \begin{tabular}[t]{lp{11cm}}  
           {\sl xdc}  & Pointer to the transducer aperture\\
           {\sl dir\_zx} & Direction (angle) in radians for the dynamic
                    focus. The direction is taken from the center for
                    the focus of the transducer in the z-x plane.\\
          {\sl dir\_zy} & Direction (angle) in radians for the dynamic
                    focus. The direction is taken from the center for
                    the focus of the transducer in the z-y plane.\\
          {\sl line\_no} & Number of line. If skipped, {\sl line\_no} is assumed
                    to be equal to '1'.
           \end{tabular} \\
  OUTPUT: & None 
\end{tabular}  

%%%%%%%%%%%%%%%%%%%%%%%%%%%%%%%%%%%55
\headline{bft\_end}
%%tth:\vspace{2cm}
%%tth:\begin{html}<hr>\end{html}
%%tth:\subsection*{bft\_end}
%%tth:\begin{html}<hr>\end{html}
\funlnk{bft_end}

Release all resources allocated by the beamforming toolbox.

\begin{tabular}[t]{lp{14cm}}  
 
 USAGE:& {\tt bft\_end}\\
 INPUT:& None\\
 OUTPUT:& None\\
\end{tabular}



%%%%%%%%%%%%%%%%%%%%%%%%%%%%%%%%%%%%%
\headline{bft\_focus}
%%tth:\vspace{2cm}
%%tth:\begin{html}<hr>\end{html}
%%tth:\subsection*{bft\_focus}
%%tth:\begin{html}<hr>\end{html}
\funlnk{bft_focus}

Create a focus time line.

\begin{tabular}[t]{lp{14cm}}  

  USAGE: & {\tt bft\_focus(xdc, times, points, line\_no)}\\
 
  INPUT:& \begin{tabular}[t]{lp{11cm}}  
           {\sl xdc}    & Pointer to aperture.\\
           {\sl times}  &  Time after which the associated focus is valid \\
           {\sl points} & Focus points. Vector with three columns (x,y,z) 
                           and one row for each field point. \\
           {\sl line\_no} & Number of line for which we set the focus. If 
                     skipped, {\sl line\_no} is assumed equal to '1'.\\
          \end{tabular} \\
  OUTPUT: & none
\end{tabular}

%%%%%%%%%%%%%%%%%%%%%%%%%%%%%%%%
\headline{bft\_focus\_times}
%%tth:\vspace{2cm}
%%tth:\begin{html}<hr>\end{html}
%%tth:\subsection*{bft\_focus\_times}
%%tth:\begin{html}<hr>\end{html}
\funlnk{bft_focus_times}

 Create a focus time line.
    The user supplies the delay times for each element.

\begin{tabular}[t]{lp{14cm}}  
  USAGE:& {\tt bft\_focus\_times(xdc, times, delays, line\_no)}\\
 
  INPUT:&  \begin{tabular}[t]{lp{11cm}}  
           {\sl xdc}  & Pointer to a transducer aperture. \\
           {\sl times} & Time after which the associated delay is valid. \\
           {\sl delays} & Delay values. Matrix with one row for each
                    time value and a number of columns equal to the
                    number of physical elements in the aperture. \\
           {\sl line\_no} & Number of line. If skipped, {\sl line\_no} is 
                          assumed to be equal to 1.
          \end{tabular}\\
  OUTPUT: & None
\end{tabular} 

%%%%%%%%%%%%%%%%%%%%%%%%%%%%%%%%%%%%%%%%%%%
\headline{bft\_free\_xdc}
%%tth:\vspace{2cm}
%%tth:\begin{html}<hr>\end{html}
%%tth:\subsection*{bft\_free\_xdc}
%%tth:\begin{html}<hr>\end{html}
\funlnk{bft_free_xdc}

Free the memory allocated for a transducer definition

\begin{tabular}[t]{lp{14cm}}  
  USAGE:& {\tt bft\_free\_xdc(xdc)}\\
  INPUT:& {\sl xdc} Pointer to the memory location returned by the
                 function \hyperlink{bft_transducer}{bft\_transducer}\\
  OUTPUT:& Nothing
\end{tabular}


%%%%%%%%%%%%%%%%%%%%%%%%%%%%%%%5
\headline{bft\_init}
%%tth:\vspace{2cm}
%%tth:\begin{html}<hr>\end{html}
%%tth:\subsection*{bft\_init}
%%tth:\begin{html}<hr>\end{html}
\funlnk{bft_init}

Initialize the BeamForming Toolbox. This command must be
    executed first in order to set some parameters and allocate the 
    the necessary memory
    
\begin{tabular}[t]{lp{14cm}}  
USAGE:& {\tt bft\_init}\\
INPUT:& None \\
OUTPUT:& None
\end{tabular}


%%%%%%%%%%%%%%%%%%%%%%%%%%%%%%%5
\headline{bft\_linear\_array}
%%tth:\vspace{2cm}
%%tth:\begin{html}<hr>\end{html}
%%tth:\subsection*{bft\_linear\_array}
%%tth:\begin{html}<hr>\end{html}
\funlnk{bft_linear_array}


 Create a linear array aperture.
 
\begin{tabular}[t]{lp{14cm}}  
 USAGE: & {\tt xdc = bft\_linear\_array(no\_elements, width, kerf)} \\
   &     {\tt xdc = bft\_linear\_array(no\_elements, pitch)} \\
 
 INPUT:& \begin{tabular}[t]{lp{11cm}}  
          {\sl no\_elements} & Number of elelements in the array \\
          {\sl pitch} & Distance between the centers of two elements [m] \\
          {\sl width} & Width in x-direction                         [m]\\
          {\sl kerf}  & Distance between two elements                [m] 
         \end{tabular} \\
          
        & The function assumes that {\sl kerf + width = pitch} \\
 OUTPUT:& {\sl xdc} - Pointer to the allocated aperture.
\end{tabular}

%%%%%%%%%%%%%%%%%%%%%%%%%%%%%%%5
\headline{bft\_no\_lines}
%%tth:\vspace{2cm}
%%tth:\begin{html}<hr>\end{html}
%%tth:\subsection*{bft\_no\_lines}
%%tth:\begin{html}<hr>\end{html}
\funlnk{bft_no_lines}

Set the number of lines that will be beamformed in parallel.
  After calling \hyperlink{bft_init}{bft\_init}, the number of lines 
that are beamformed in   parallel is 1. If the user wants to beamform
a whole image in one  command, he/she must set the number of lines,
and then specify the focal zones for each of the lines.

\begin{tabular}[t]{lp{14cm}}  
 
 USAGE: & {\tt bft\_no\_lines(no\_lines)}\\
 INPUT:& {\sl no\_lines}  ~~Number of lines beamformed in parallel \\
 OUTPUT: & None
\end{tabular}


%%%%%%%%%%%%%%%%%%%%%%%%%%%%%%%5
\headline{bft\_param}
%%tth:\vspace{2cm}
%%tth:\begin{html}<hr>\end{html} 
%%tth:\subsection*{bft\_param}
%%tth:\begin{html}<hr>\end{html}
\funlnk{bft_param}

Set a parameter of the BeamForming Toolbox

\begin{tabular}[t]{lp{14cm}}  

 USAGE: & {\tt bft\_param(name, value)}\\
 
 INPUT: & \begin{tabular}[t]{lp{11cm}}  
          {\sl name} & Name of  the parameter (string). Currently supported:\\
        &  \begin{tabular}[t]{c|c|c|c}
           \hline 
            name &        Meaning       & Default value &  Unit \\
            \hline 
                  'c' & Speed of sound.       & 1540         &  m/s \\
                  'fs'& Sampling frequency    & 40,000,000   &  Hz \\
            \hline       
          \end{tabular} \\\\
         
          {\sl value} & New value for the parameter. Must be scalar. 
          \end{tabular} \\
 OUTPUT: & None
\end{tabular}


%%%%%%%%%%%%%%%%%%%%%%%%%%%%%%%5
\headline{bft\_sub\_image}
%%tth:\vspace{2cm}
%%tth:\begin{html}<hr>\end{html} 
%%tth:\subsection*{bft\_sub\_image}
%%tth:\begin{html}<hr>\end{html}
\funlnk{bft_sub_image}

Subtract one low-res image from  high-res one.

\begin{tabular}[t]{lp{14cm}}  
 
 USAGE: & {\tt [hi\_res] = bft\_sub\_image(hi\_res, lo\_res, element, start\_time)} \\
 
 INPUT:& \begin{tabular}[t]{lp{11cm}}  
          {\sl hi\_res}  & High resolution RF image. One column per scan line.\\
          {\sl lo\_res}  & Low resolution RF image. One column per scan line.\\
          {\sl element} & Number of element, used to acquire the low resolution
                    image. \\
          {\sl start\_time}    & Arrival time of the first sample of the RF lines.
          \end{tabular} \\
 OUTPUT:& hi\_res - The high resolution image.
\end{tabular}

%%%%%%%%%%%%%%%%%%%%%%%%%%%%%%%5
\headline{bft\_sum\_apodization}
%%tth:\vspace{2cm}
%%tth:\begin{html}<hr>\end{html} 
%%tth:\subsection*{bft\_sum\_apodization}
%%tth:\begin{html}<hr>\end{html}
\funlnk{bft_sum_apodization}

Create a summation apodization time line.
    This function is used in the case that the individual low
    resolution images must be weighted during the summation

\begin{tabular}[t]{lp{14cm}}  
 
  USAGE: & {\tt  bft\_sum\_apodization(xdc, times, values, line\_no)} \\
 
  INPUT: & \begin{tabular}[t]{lp{14cm}}  
          {\sl xdc} & Pointer to a transducer aperture.\\
          {\sl times} & Time after which the associated apodization is valid.\\
          {\sl values} & Apodization values. Matrix with one row for each
                   time value and a number of columns equal to the 
                   number of physical elements in the aperture. \\
          {\sl line\_no} & Number of line. If skipped, {\sl line\_no} is assumed
                    to be equal to '1'.
         \end{tabular}\\
  OUTPUT: & None         
\end{tabular}


\headline{bft\_sum\_images}
%%tth:\vspace{2cm}
%%tth:\begin{html}<hr>\end{html} 
%%tth:\subsection*{bft\_sum\_images}
%%tth:\begin{html}<hr>\end{html}
\funlnk{bft_sum_images}

 Sum 2 low resolution images in 1 high resolution.

\begin{tabular}[t]{lp{14cm}}  
 
 USAGE  : {\tt [hi\_res] = bft\_sum\_images(image1, ele1, image2, ele2, time) }\\
 
 INPUT  : \begin{tabular}[t]{lp{141cm}}  
          {\sl image1} & Matrix with the RF data for the image. The number
                   of columns corresponds to the number of lines \\
          {\sl ele1}   & Number of emitting element used to obtain the image.\\ 
          {\sl image2} & Matrix with the RF data for the image. The number
                   of columns corresponding to the number of lines \\
          {\sl ele2}   & Number of emitting element used to obtain the image.\\
          {\sl time}   &  The arrival time of the first samples. The two images
                    must be aligned in time 
          \end{tabular}
 OUTPUT : hi\_res - Higher resolution image
\end{tabular}

\headline{bft\_transducer}
%%tth:\vspace{2cm}
%%tth:\begin{html}<hr>\end{html} 
%%tth:\subsection*{bft\_transducer}
%%tth:\begin{html}<hr>\end{html}
\funlnk{bft_transducer}
 
 Create a new transducer definition. 
    The transducer definition is necessary for the calculation of 
    the delays.

\begin{tabular}[t]{lp{14cm}}  
 
 USAGE: &  {\tt xdc = bft\_transducer(centers)}\\
 
 INPUT: & \begin{tabular}[t]{lp{11cm}}  
            {\sl centers} & Matrix with the coordinates of the centers of 
                   the elements. It has 3 columns (x,y,z) and a
                   number of rows equal to the number of elements.
                   The coordinates are specified in [m]
         \end{tabular}\\ 
 OUTPUT: & \begin{tabular}[t]{lp{11cm}}  
          xdc & Pointer to the memory location with the transducer 
               definition. Do not alter this value !!!
          \end{tabular}
\end{tabular}

%%tth:\begin{html}<hr>\end{html}
\chapter{Examples}
\label{chap_examples}
%%tth:\begin{html}<hr>\end{html}

\section{Using Field II simulations}

\headline{Phased array B-mode image}
%%tth:\vspace{2cm}
%%tth:\begin{html}<hr>\end{html} 
%%tth:\subsection*{phased array B-mode image}
%%tth:\begin{html}<hr>\end{html}
\funlnk{phased_bmode}

{\footnotesize\noindent
1{\tt~~~~}{\tt {\sl \%PHASED\_IMAGE~Create~phased~array~B-mode~image~with~BFT.}}\\
2{\tt~~~~}{\tt {\sl \%~~~~This~script~creates~a~B-mode~PSF~line~by~line.~Each~line~is~}}\\
3{\tt~~~~}{\tt {\sl \%~~~~calculated~using~CALC\_SCAT~and~CALC\_SCAT\_MULTY.~The~rf\_data}}\\
4{\tt~~~~}{\tt {\sl \%~~~~from~CALC\_SCAT\_MULTI~is~passed~to~~the~beamforming~toolbox,}}\\
5{\tt~~~~}{\tt {\sl \%~~~~and~in~the~end~the~results~are~compared.}}\\
6{\tt~~~~}{\tt {\sl \%}}\\
7{\tt~~~~}{\tt {\sl \%~~~~The~function~calls~XDC\_FOCUS,~and~BFT\_FOCUS~in~order~to~set~the}}\\
8{\tt~~~~}{\tt {\sl \%~~~~the~delays.~}}\\
9{\tt~~~~}{\tt }\\
10{\tt~~~}{\tt {\sl \%~VERSION~1.0,~29~Feb.~2000,~Svetoslav~Nikolov}}\\
11{\tt~~~}{\tt }\\
12{\tt~~~}{\tt f0~=~4e6;~~~~~~~~~~~~~~{\sl \%~~Central~frequency~~~~~~~~~~~~~~~~~~~~~~~[Hz]}}\\
13{\tt~~~}{\tt fs~=~100e6;~~~~~~~~~~~~{\sl \%~~Sampling~frequency~~~~~~~~~~~~~~~~~~~~~~[Hz]}}\\
14{\tt~~~}{\tt c~=~1540;~~~~~~~~~~~~~~{\sl \%~~Speed~of~sound~~~~~~~~~~~~~~~~~~~~~~~~~~[m/s]}}\\
15{\tt~~~}{\tt no\_elements~=~64;~~~~~~{\sl \%~~Number~of~elements~in~the~transducer}}\\
16{\tt~~~}{\tt }\\
17{\tt~~~}{\tt lambda~=~c~/~f0;~~~~~~~{\sl \%~Wavelength~~~~~~~~~~~~~~~~~~~~~~~~~~~~~~~[m]}}\\
18{\tt~~~}{\tt pitch~=~lambda~/~2;~~~~{\sl \%~Pitch~-~center-to-center~~~~~~~~~~~~~~~~~[m]}}\\
19{\tt~~~}{\tt width~=~.95*pitch;~~~~~{\sl \%~Width~of~the~element~~~~~~~~~~~~~~~~~~~~~[m]}}\\
20{\tt~~~}{\tt kerf~=~pitch~-~width;~~{\sl \%~Inter-element~spacing~~~~~~~~~~~~~~~~~~~~[m]}}\\
21{\tt~~~}{\tt height~=~10/1000;~~~~~~{\sl \%~Size~in~the~Y~direction~~~~~~~~~~~~~~~~~~[m]}}\\
22{\tt~~~}{\tt ~}\\
23{\tt~~~}{\tt ~}\\
24{\tt~~~}{\tt {\sl \%~~Define~the~impulse~response~of~the~transducer}}\\
25{\tt~~~}{\tt impulse\_response~=~sin(2*pi*f0*(0:1/fs:2/f0));}\\
26{\tt~~~}{\tt impulse\_response~=~impulse\_response.*hanning(length(impulse\_response))';}\\
27{\tt~~~}{\tt excitation~=~impulse\_response;}\\
28{\tt~~~}{\tt }\\
29{\tt~~~}{\tt {\sl \%~~Define~the~phantom}}\\
30{\tt~~~}{\tt }\\
31{\tt~~~}{\tt pht\_pos~=~[0~0~20;}\\
32{\tt~~~}{\tt ~~~~~~~~~~~0~0~30;}\\
33{\tt~~~}{\tt ~~~~~~~~~~~0~0~40;}\\
34{\tt~~~}{\tt ~~~~~~~~~~~0~0~50;}\\
35{\tt~~~}{\tt ~~~~~~~~~~~0~0~60;}\\
36{\tt~~~}{\tt ~~~~~~~~~~~0~0~70;}\\
37{\tt~~~}{\tt ~~~~~~~~~~~0~0~80;}\\
38{\tt~~~}{\tt ~~~~~~~~~~~0~0~90;}\\
39{\tt~~~}{\tt ~~~~~~~~~~~]~/~1000;~~~~~~~~~{\sl \%~~The~position~of~the~phantom}}\\
40{\tt~~~}{\tt pht\_amp~=~20*ones(8,1);~~~~~~{\sl \%~~The~amplitude~of~the~back-scatter}}\\
41{\tt~~~}{\tt }\\
42{\tt~~~}{\tt }\\
43{\tt~~~}{\tt }\\
44{\tt~~~}{\tt {\sl \%~~Define~the~focus~}}\\
45{\tt~~~}{\tt focus\_r~=~[20;30;40;50;60;70;80;90]~/~1000;}\\
46{\tt~~~}{\tt T~=~(focus\_r-5/1000)/c~*2;}\\
47{\tt~~~}{\tt }\\
48{\tt~~~}{\tt }\\
49{\tt~~~}{\tt {\sl \%~~Initialize~the~program}}\\
50{\tt~~~}{\tt field\_init(0);}\\
51{\tt~~~}{\tt bft\_init;}\\
52{\tt~~~}{\tt }\\
53{\tt~~~}{\tt }\\
54{\tt~~~}{\tt {\sl \%~~Set~some~paramters}}\\
55{\tt~~~}{\tt set\_field('c',~c);}\\
56{\tt~~~}{\tt bft\_param('c',~c);}\\
57{\tt~~~}{\tt }\\
58{\tt~~~}{\tt set\_field('fs',~fs);}\\
59{\tt~~~}{\tt bft\_param('fs',~fs);}\\
60{\tt~~~}{\tt }\\
61{\tt~~~}{\tt }\\
62{\tt~~~}{\tt }\\
63{\tt~~~}{\tt {\sl \%~Create~some~apertures.}}\\
64{\tt~~~}{\tt }\\
65{\tt~~~}{\tt xmt~=~xdc\_linear\_array(no\_elements,width,height,kerf,1,1,[0~0~0]);}\\
66{\tt~~~}{\tt rcv~=~xdc\_linear\_array(no\_elements,width,height,kerf,1,1,[0~0~0]);}\\
67{\tt~~~}{\tt }\\
68{\tt~~~}{\tt xdc~=~bft\_linear\_array(no\_elements,~width,~kerf);}\\
69{\tt~~~}{\tt }\\
70{\tt~~~}{\tt }\\
71{\tt~~~}{\tt {\sl \%~Set~the~impulse~responses}}\\
72{\tt~~~}{\tt xdc\_impulse(rcv,~impulse\_response);}\\
73{\tt~~~}{\tt xdc\_impulse(xmt,~impulse\_response);}\\
74{\tt~~~}{\tt }\\
75{\tt~~~}{\tt xdc\_excitation(xmt,~excitation);}\\
76{\tt~~~}{\tt }\\
77{\tt~~~}{\tt }\\
78{\tt~~~}{\tt }\\
79{\tt~~~}{\tt {\sl \%~~Define~and~create~the~image}}\\
80{\tt~~~}{\tt sector~=~30~*~pi~/~180;}\\
81{\tt~~~}{\tt no\_lines~=~32;}\\
82{\tt~~~}{\tt d\_theta~=~sector~/~(no\_lines-1);}\\
83{\tt~~~}{\tt theta~=~-(no\_lines-1)~/~2~*~d\_theta;}\\
84{\tt~~~}{\tt }\\
85{\tt~~~}{\tt Rmax~=~max(sqrt(pht\_pos(:,1).\^~2~+~pht\_pos(:,2).\^~2~~+~pht\_pos(:,3).\^~2))~+~15/1000;}\\
86{\tt~~~}{\tt }\\
87{\tt~~~}{\tt no\_rf\_samples~=~ceil(2*Rmax/c~*~fs);}\\
88{\tt~~~}{\tt rf\_line~=~zeros(no\_rf\_samples,~1);}\\
89{\tt~~~}{\tt bf\_line~=~zeros(no\_rf\_samples,~1);}\\
90{\tt~~~}{\tt }\\
91{\tt~~~}{\tt env\_line~=~zeros(no\_rf\_samples,~no\_lines);}\\
92{\tt~~~}{\tt env\_bf~=~~zeros(no\_rf\_samples,~no\_lines);}\\
93{\tt~~~}{\tt }\\
94{\tt~~~}{\tt }\\
95{\tt~~~}{\tt xmt\_r~=~(max(focus\_r)~+~min(focus\_r)~)/2;}\\
96{\tt~~~}{\tt }\\
97{\tt~~~}{\tt for~i~=~1~:~no\_lines}\\
98{\tt~~~}{\tt ~~rf\_line(:)~=~0;~~}\\
99{\tt~~~}{\tt ~~disp(['Line~no~'~num2str(i)])}\\
100{\tt~~}{\tt ~~}\\
101{\tt~~}{\tt ~~focus~=~[sin(theta)*focus\_r,~zeros(length(focus\_r),1),~cos(theta)*focus\_r];}\\
102{\tt~~}{\tt ~~xmt\_f~=~[sin(theta)*xmt\_r,~zeros(length(xmt\_r),1),~cos(theta)*xmt\_r];}\\
103{\tt~~}{\tt ~~xdc\_center\_focus(xmt,[0~0~0])}\\
104{\tt~~}{\tt ~~xdc\_center\_focus(rcv,[0~0~0])}\\
105{\tt~~}{\tt ~~bft\_center\_focus([0~0~0]);}\\
106{\tt~~}{\tt ~~}\\
107{\tt~~}{\tt ~~xdc\_focus(xmt,~0,~xmt\_f);}\\
108{\tt~~}{\tt ~~xdc\_focus(rcv,~T,~focus);}\\
109{\tt~~}{\tt ~~bft\_focus(xdc,~T,~focus);}\\
110{\tt~~}{\tt ~~}\\
111{\tt~~}{\tt ~~{\sl \%~~Beamform~with~Field~II}}\\
112{\tt~~}{\tt ~~[rf\_temp,~t(i)]~=~calc\_scat(xmt,rcv,~pht\_pos,~pht\_amp);}\\
113{\tt~~}{\tt ~~}\\
114{\tt~~}{\tt ~~{\sl \%~~Beamform~with~BFT}}\\
115{\tt~~}{\tt ~~xdc\_focus\_times(rcv,~0,~zeros(1,no\_elements));}\\
116{\tt~~}{\tt ~~[rf\_data,~start\_t]~=~calc\_scat\_multi(xmt,rcv,~pht\_pos,~pht\_amp);}\\
117{\tt~~}{\tt ~~rf\_data~=~[zeros(300,no\_elements);~rf\_data;~zeros(300,no\_elements)];}\\
118{\tt~~}{\tt }\\
119{\tt~~}{\tt ~~start\_t~=~start\_t~-~300~~/~fs;}\\
120{\tt~~}{\tt ~~bf\_temp~=~bft\_beamform(start\_t,~rf\_data);}\\
121{\tt~~}{\tt ~~}\\
122{\tt~~}{\tt ~~start\_sample~=~t(i)*fs;~no\_temp\_samples~=~length(rf\_temp);}\\
123{\tt~~}{\tt ~~}\\
124{\tt~~}{\tt ~~rf\_line(start\_sample:start\_sample+no\_temp\_samples-1)~=~rf\_temp(1:no\_temp\_samples);}\\
125{\tt~~}{\tt ~~env\_line(:,i)~=~abs(hilbert(rf\_line(:)));}\\
126{\tt~~}{\tt }\\
127{\tt~~}{\tt ~~start\_sample~=~floor(start\_t*fs);~no\_temp\_samples~=~length(bf\_temp);}\\
128{\tt~~}{\tt ~~bf\_line(start\_sample:start\_sample+no\_temp\_samples-1)~=~bf\_temp(1:no\_temp\_samples);}\\
129{\tt~~}{\tt ~~env\_bf(:,i)~=~abs(hilbert(bf\_line(:)));}\\
130{\tt~~}{\tt ~~theta~=~theta~+~d\_theta;}\\
131{\tt~~}{\tt }\\
132{\tt~~}{\tt end}\\
133{\tt~~}{\tt }\\
134{\tt~~}{\tt {\sl \%~~Release~the~allocated~memory}}\\
135{\tt~~}{\tt }\\
136{\tt~~}{\tt field\_end}\\
137{\tt~~}{\tt bft\_end}\\
138{\tt~~}{\tt }\\
139{\tt~~}{\tt env\_line~=~env\_line~/~max(max(abs(env\_line)));}\\
140{\tt~~}{\tt env\_bf~=~env\_bf~/~max(max(abs(env\_bf)));}\\
141{\tt~~}{\tt }\\
142{\tt~~}{\tt figure;}\\
143{\tt~~}{\tt subplot(1,2,1)}\\
144{\tt~~}{\tt imagesc([-sector/2~sector/2]*180/pi,[0~Rmax]*1000,20*log10(env\_line~+~0.001))}\\
145{\tt~~}{\tt axis('image')}\\
146{\tt~~}{\tt xlabel('Angle~[deg]');}\\
147{\tt~~}{\tt ylabel('Axial~distance~[mm]')}\\
148{\tt~~}{\tt title('Beamformed~by~Field~II~');}\\
149{\tt~~}{\tt }\\
150{\tt~~}{\tt subplot(1,2,2)}\\
151{\tt~~}{\tt imagesc([-sector/2~sector/2]*180/pi,[0~Rmax]*1000,20*log10(env\_bf~+~0.001));}\\
152{\tt~~}{\tt title('Beamformed~by~BFT');}\\
153{\tt~~}{\tt xlabel('Angle~[deg]');}\\
154{\tt~~}{\tt ylabel('Axial~distance~[mm]')}\\
155{\tt~~}{\tt axis('image')}\\
156{\tt~~}{\tt }\\
157{\tt~~}{\tt colorbar}\\
158{\tt~~}{\tt colormap(gray)}\\
159{\tt~~}{\tt }\\
160{\tt~~}{\tt clc}\\
161{\tt~~}{\tt disp(['~'~10~10~10~10~]);}\\
162{\tt~~}{\tt disp([9~'*****************************************************']);}\\
163{\tt~~}{\tt disp([9~'*~~~~~~~~~~~~~~~~~~~~~~~~~~~~~~~~~~~~~~~~~~~~~~~~~~~*']);}\\
164{\tt~~}{\tt disp([9~'*~The~image~beamformed~by~Field~II~is~in~"env\_line"~*']);}\\
165{\tt~~}{\tt disp([9~'*~The~image~beamformed~by~BFT~is~in~"env\_bf"~~~~~~~~*']);}\\
166{\tt~~}{\tt disp([9~'*~~~~~~~~~~~~~~~~~~~~~~~~~~~~~~~~~~~~~~~~~~~~~~~~~~~*'])}\\
167{\tt~~}{\tt disp([9~'*****************************************************']);}\\
168{\tt~~}{\tt disp(['~'~10~10~]);}\\
169{\tt~~}{\tt }\\
170{\tt~~}{\tt }\\
}

%%%%%%%%%%%%%%%%%%%%%%%%%%%%%%%%%%%%%%%
\headline{Dynamic focusing}
%%tth:\vspace{2cm}
%%tth:\begin{html}<hr>\end{html} 
%%tth:\subsection*{Dynamic focusing}
%%tth:\begin{html}<hr>\end{html}
\funlnk{dynamic_focusing}

{\footnotesize\noindent
1{\tt~~~~}{\tt {\sl \%PHASED\_DYN\_IMAGE~Create~a~phased-array~B-mode~image,~using~the~}}\\
2{\tt~~~~}{\tt {\sl \%~~~~commands~for~setting~a~dynamic~focusing}}\\
3{\tt~~~~}{\tt }\\
4{\tt~~~~}{\tt {\sl \%VERSION~1.0,~29~Feb~2000,~Svetoslav~Nikolov}}\\
5{\tt~~~~}{\tt }\\
6{\tt~~~~}{\tt f0~=~4e6;~~~~~~~~~~~~~~{\sl \%~~Central~frequency~~~~~~~~~~~~~~~~~~~~~~~~[Hz]}}\\
7{\tt~~~~}{\tt fs~=~100e6;~~~~~~~~~~~~{\sl \%~~Sampling~frequency~~~~~~~~~~~~~~~~~~~~~~~[Hz]}}\\
8{\tt~~~~}{\tt c~=~1540;~~~~~~~~~~~~~~{\sl \%~~Speed~of~sound~~~~~~~~~~~~~~~~~~~~~~~~~~~[m/s]}}\\
9{\tt~~~~}{\tt B~=~.35;~~~~~~~~~~~~~~~{\sl \%~~Relative~bandwith~~~~~~~~~~~~~~~~~~~~~~~~[fraction]}}\\
10{\tt~~~}{\tt no\_elements~=~64;~~~~~~{\sl \%~~Number~of~elements~in~the~transducer~~~~~}}\\
11{\tt~~~}{\tt }\\
12{\tt~~~}{\tt lambda~=~c~/~f0;~~~~~~~{\sl \%~Wavelength~~~~~~~~~~~~~~~~~~~~~~~~~~~~~~~~[m]}}\\
13{\tt~~~}{\tt pitch~=~lambda~/~2;~~~~{\sl \%~Pitch~-~center-to-center~~~~~~~~~~~~~~~~~~[m]}}\\
14{\tt~~~}{\tt width~=~.95*pitch;~~~~~{\sl \%~Width~of~the~element~~~~~~~~~~~~~~~~~~~~~~[m]}}\\
15{\tt~~~}{\tt kerf~=~pitch~-~width;~~{\sl \%~Inter-element~spacing~~~~~~~~~~~~~~~~~~~~~[m]}}\\
16{\tt~~~}{\tt height~=~10/1000;~~~~~~{\sl \%~Size~in~the~Y~direction~~~~~~~~~~~~~~~~~~~[m]}}\\
17{\tt~~~}{\tt ~}\\
18{\tt~~~}{\tt ~}\\
19{\tt~~~}{\tt {\sl \%~~Define~the~impulse~response~of~the~transducer}}\\
20{\tt~~~}{\tt impulse\_response~=~sin(2*pi*f0*(0:1/fs:2/f0));}\\
21{\tt~~~}{\tt impulse\_response~=~impulse\_response.*hanning(length(impulse\_response))';}\\
22{\tt~~~}{\tt excitation~=~impulse\_response;}\\
23{\tt~~~}{\tt }\\
24{\tt~~~}{\tt {\sl \%~~Define~the~phantom}}\\
25{\tt~~~}{\tt }\\
26{\tt~~~}{\tt pht\_pos~=~[0~0~20;}\\
27{\tt~~~}{\tt ~~~~~~~~~~~0~0~30;}\\
28{\tt~~~}{\tt ~~~~~~~~~~~0~0~40;}\\
29{\tt~~~}{\tt ~~~~~~~~~~~0~0~50;}\\
30{\tt~~~}{\tt ~~~~~~~~~~~0~0~60;}\\
31{\tt~~~}{\tt ~~~~~~~~~~~0~0~70;}\\
32{\tt~~~}{\tt ~~~~~~~~~~~0~0~80;]~/~1000;~~~~~~~~~{\sl \%~~The~position~of~the~phantom}}\\
33{\tt~~~}{\tt pht\_amp~=~20*ones(7,1);~~~~~~{\sl \%~~The~amplitude~of~the~back-scatter}}\\
34{\tt~~~}{\tt }\\
35{\tt~~~}{\tt {\sl \%~~Define~the~focus~}}\\
36{\tt~~~}{\tt focus\_r~=~[20;30;40;50;60;70;80;90]~/~1000;}\\
37{\tt~~~}{\tt T~=~(focus\_r-5/1000)/c~*2;}\\
38{\tt~~~}{\tt }\\
39{\tt~~~}{\tt {\sl \%~~Initialize~the~program}}\\
40{\tt~~~}{\tt field\_init(0);}\\
41{\tt~~~}{\tt bft\_init;}\\
42{\tt~~~}{\tt }\\
43{\tt~~~}{\tt {\sl \%~~Set~some~paramters}}\\
44{\tt~~~}{\tt set\_field('c',~c);}\\
45{\tt~~~}{\tt bft\_param('c',~c);}\\
46{\tt~~~}{\tt }\\
47{\tt~~~}{\tt set\_field('fs',~fs);}\\
48{\tt~~~}{\tt bft\_param('fs',~fs);}\\
49{\tt~~~}{\tt }\\
50{\tt~~~}{\tt }\\
51{\tt~~~}{\tt {\sl \%~Create~some~apertures.}}\\
52{\tt~~~}{\tt }\\
53{\tt~~~}{\tt xmt~=~xdc\_linear\_array(no\_elements,width,height,kerf,1,1,[0~0~0]);}\\
54{\tt~~~}{\tt rcv~=~xdc\_linear\_array(no\_elements,width,height,kerf,1,1,[0~0~0]);}\\
55{\tt~~~}{\tt }\\
56{\tt~~~}{\tt xdc~=~bft\_linear\_array(no\_elements,~width,~kerf);}\\
57{\tt~~~}{\tt }\\
58{\tt~~~}{\tt }\\
59{\tt~~~}{\tt {\sl \%~Set~the~impulse~responses}}\\
60{\tt~~~}{\tt xdc\_impulse(rcv,~impulse\_response);}\\
61{\tt~~~}{\tt xdc\_impulse(xmt,~impulse\_response);}\\
62{\tt~~~}{\tt }\\
63{\tt~~~}{\tt xdc\_excitation(xmt,~excitation);}\\
64{\tt~~~}{\tt }\\
65{\tt~~~}{\tt }\\
66{\tt~~~}{\tt {\sl \%~Set~the~apodization}}\\
67{\tt~~~}{\tt }\\
68{\tt~~~}{\tt xdc\_apodization(xmt,~0,~ones(1,no\_elements))}\\
69{\tt~~~}{\tt xdc\_apodization(rcv,~0,~ones(1,no\_elements))}\\
70{\tt~~~}{\tt bft\_apodization(xdc,~0~,~ones(1,no\_elements))}\\
71{\tt~~~}{\tt }\\
72{\tt~~~}{\tt {\sl \%~~Define~and~create~the~image}}\\
73{\tt~~~}{\tt sector~=~30~*~pi~/~180;}\\
74{\tt~~~}{\tt no\_lines~=~32;}\\
75{\tt~~~}{\tt d\_theta~=~sector~/~(no\_lines-1);}\\
76{\tt~~~}{\tt theta~=~-(no\_lines-1)~/~2~*~d\_theta;}\\
77{\tt~~~}{\tt }\\
78{\tt~~~}{\tt Rmax~=~max(sqrt(pht\_pos(:,1).\^~2~+~pht\_pos(:,2).\^~2~~+~pht\_pos(:,3).\^~2))~+~15/1000;}\\
79{\tt~~~}{\tt }\\
80{\tt~~~}{\tt no\_rf\_samples~=~ceil(2*Rmax/c~*~fs);}\\
81{\tt~~~}{\tt rf\_line~=~zeros(no\_rf\_samples,~1);}\\
82{\tt~~~}{\tt bf\_line~=~zeros(no\_rf\_samples,~1);}\\
83{\tt~~~}{\tt }\\
84{\tt~~~}{\tt env\_line~=~zeros(no\_rf\_samples,~no\_lines);}\\
85{\tt~~~}{\tt env\_bf~=~~zeros(no\_rf\_samples,~no\_lines);}\\
86{\tt~~~}{\tt }\\
87{\tt~~~}{\tt }\\
88{\tt~~~}{\tt xmt\_r~=~(max(focus\_r)~+~min(focus\_r)~)/2;}\\
89{\tt~~~}{\tt bf~=~cell(no\_lines,1);}\\
90{\tt~~~}{\tt for~i~=~1~:~no\_lines}\\
91{\tt~~~}{\tt ~~rf\_line(:)~=~0;~~}\\
92{\tt~~~}{\tt ~~theta}\\
93{\tt~~~}{\tt ~~}\\
94{\tt~~~}{\tt ~~xmt\_f~=~[sin(theta)*xmt\_r,~zeros(length(xmt\_r),1),~cos(theta)*xmt\_r];}\\
95{\tt~~~}{\tt ~~xdc\_center\_focus(xmt,[0~0~0])}\\
96{\tt~~~}{\tt ~~xdc\_center\_focus(rcv,[0~0~0])}\\
97{\tt~~~}{\tt ~~bft\_center\_focus([0~0~0]);}\\
98{\tt~~~}{\tt ~~}\\
99{\tt~~~}{\tt ~~xdc\_focus(xmt,~0,~xmt\_f);}\\
100{\tt~~}{\tt ~~xdc\_dynamic\_focus(rcv,~0,~theta,~0);}\\
101{\tt~~}{\tt ~~}\\
102{\tt~~}{\tt ~~{\sl \%~~Beamform~with~Field~II}}\\
103{\tt~~}{\tt ~~[rf\_temp,~t(i)]~=~calc\_scat(xmt,rcv,~pht\_pos,~pht\_amp);}\\
104{\tt~~}{\tt ~~}\\
105{\tt~~}{\tt ~~{\sl \%~~Beamform~with~BFT}}\\
106{\tt~~}{\tt ~~bft\_dynamic\_focus(xdc,~theta,~0)}\\
107{\tt~~}{\tt ~~xdc\_focus\_times(rcv,~0,~zeros(1,no\_elements));}\\
108{\tt~~}{\tt ~~[rf\_data,~start\_t]~=~calc\_scat\_multi(xmt,rcv,~pht\_pos,~pht\_amp);}\\
109{\tt~~}{\tt ~~}\\
110{\tt~~}{\tt ~~rf\_data~=~[zeros(300,no\_elements);~rf\_data;~zeros(300,no\_elements)];}\\
111{\tt~~}{\tt }\\
112{\tt~~}{\tt ~~start\_t~=~start\_t~-~300~~/~fs;}\\
113{\tt~~}{\tt ~~bf\_temp~=~bft\_beamform(start\_t,~rf\_data);}\\
114{\tt~~}{\tt ~~}\\
115{\tt~~}{\tt ~~start\_sample~=~t(i)*fs;~no\_temp\_samples~=~length(rf\_temp);}\\
116{\tt~~}{\tt ~~}\\
117{\tt~~}{\tt ~~rf\_line(start\_sample:start\_sample+no\_temp\_samples-1)~=~rf\_temp(1:no\_temp\_samples);}\\
118{\tt~~}{\tt ~~env\_line(:,i)~=~abs(hilbert(rf\_line(:)));}\\
119{\tt~~}{\tt }\\
120{\tt~~}{\tt ~~start\_sample~=~floor(start\_t*fs);~no\_temp\_samples~=~length(bf\_temp);}\\
121{\tt~~}{\tt ~~bf{i}~=~bf\_temp;}\\
122{\tt~~}{\tt ~~}\\
123{\tt~~}{\tt ~~bf\_line(start\_sample:start\_sample+no\_temp\_samples-1)~=~bf\_temp(1:no\_temp\_samples);}\\
124{\tt~~}{\tt ~~env\_bf(:,i)~=~abs(hilbert(bf\_line(:)));}\\
125{\tt~~}{\tt ~~theta~=~theta~+~d\_theta;}\\
126{\tt~~}{\tt }\\
127{\tt~~}{\tt end}\\
128{\tt~~}{\tt }\\
129{\tt~~}{\tt {\sl \%~~Release~the~allocated~memory}}\\
130{\tt~~}{\tt }\\
131{\tt~~}{\tt field\_end}\\
132{\tt~~}{\tt bft\_end}\\
133{\tt~~}{\tt env\_line~=~env\_line~/~max(max(abs(env\_line)));}\\
134{\tt~~}{\tt env\_bf~=~env\_bf~/~max(max(abs(env\_bf)));}\\
135{\tt~~}{\tt }\\
136{\tt~~}{\tt figure;}\\
137{\tt~~}{\tt subplot(1,2,1)}\\
138{\tt~~}{\tt imagesc([-sector/2~sector/2]*180/pi,[0~Rmax]*1000,20*log10(env\_line~+~0.001))}\\
139{\tt~~}{\tt axis('image')}\\
140{\tt~~}{\tt xlabel('Angle~[deg]');}\\
141{\tt~~}{\tt ylabel('Axial~distance~[mm]')}\\
142{\tt~~}{\tt title('Beamformed~by~Field~II~');}\\
143{\tt~~}{\tt }\\
144{\tt~~}{\tt subplot(1,2,2)}\\
145{\tt~~}{\tt imagesc([-sector/2~sector/2]*180/pi,[0~Rmax]*1000,20*log10(env\_bf~+~0.001));}\\
146{\tt~~}{\tt title('Beamformed~by~BFT');}\\
147{\tt~~}{\tt xlabel('Angle~[deg]');}\\
148{\tt~~}{\tt ylabel('Axial~distance~[mm]')}\\
149{\tt~~}{\tt axis('image')}\\
150{\tt~~}{\tt }\\
151{\tt~~}{\tt colorbar}\\
152{\tt~~}{\tt colormap(gray)}\\
153{\tt~~}{\tt }\\
154{\tt~~}{\tt clc}\\
155{\tt~~}{\tt disp(['~'~10~10~10~10~]);}\\
156{\tt~~}{\tt disp([9~'*****************************************************']);}\\
157{\tt~~}{\tt disp([9~'*~~~~~~~~~~~~~~~~~~~~~~~~~~~~~~~~~~~~~~~~~~~~~~~~~~~*']);}\\
158{\tt~~}{\tt disp([9~'*~The~image~beamformed~by~Field~II~is~in~"env\_line"~*']);}\\
159{\tt~~}{\tt disp([9~'*~The~image~beamformed~by~BFT~is~in~"env\_bf"~~~~~~~~*']);}\\
160{\tt~~}{\tt disp([9~'*~~~~~~~~~~~~~~~~~~~~~~~~~~~~~~~~~~~~~~~~~~~~~~~~~~~*'])}\\
161{\tt~~}{\tt disp([9~'*****************************************************']);}\\
162{\tt~~}{\tt disp(['~'~10~10~]);}\\
163{\tt~~}{\tt }\\
164{\tt~~}{\tt }\\
}

\headline{Synthetic Aperture Focusing}
%%tth:\vspace{2cm}
%%tth:\begin{html}<hr>\end{html} 
%%tth:\subsection*{Synthetic Aperture Focusing}
%%tth:\begin{html}<hr>\end{html}
\funlnk{sar_focusing}

{\footnotesize\noindent
1{\tt~~~~}{\tt {\sl \%SYNTHETIC~Synthetic~aperture~beamforming~with~BFT}}\\
2{\tt~~~~}{\tt {\sl \%~}}\\
3{\tt~~~~}{\tt }\\
4{\tt~~~~}{\tt }\\
5{\tt~~~~}{\tt f0~=~4e6;~~~~~~~~~~~~~~{\sl \%~~Central~frequency}}\\
6{\tt~~~~}{\tt fs~=~100e6;~~~~~~~~~~~~{\sl \%~~Sampling~frequency}}\\
7{\tt~~~~}{\tt c~=~1540;~~~~~~~~~~~~~~{\sl \%~~Speed~of~sound}}\\
8{\tt~~~~}{\tt no\_elements~=~64;~~~~~~{\sl \%~~Number~of~elements~in~the~transducer}}\\
9{\tt~~~~}{\tt }\\
10{\tt~~~}{\tt lambda~=~c~/~f0;~~~~~~~{\sl \%~Wavelength}}\\
11{\tt~~~}{\tt pitch~=~lambda~/~2;~~~~{\sl \%~Pitch~-~center-to-center}}\\
12{\tt~~~}{\tt width~=~.95*pitch;~~~~~{\sl \%~Width~of~the~element}}\\
13{\tt~~~}{\tt kerf~=~pitch~-~width;~~{\sl \%~Inter-element~spacing}}\\
14{\tt~~~}{\tt height~=~10/1000;~~~~~~{\sl \%~Size~in~the~Y~direction}}\\
15{\tt~~~}{\tt ~}\\
16{\tt~~~}{\tt ~}\\
17{\tt~~~}{\tt {\sl \%~~Define~the~impulse~response~of~the~transducer}}\\
18{\tt~~~}{\tt impulse\_response~=~sin(3*pi*f0*(0:1/fs:2/f0));}\\
19{\tt~~~}{\tt impulse\_response~=~impulse\_response.*hanning(length(impulse\_response))';}\\
20{\tt~~~}{\tt excitation~=~sin(2*pi*f0*(0:1/fs:3/f0));}\\
21{\tt~~~}{\tt }\\
22{\tt~~~}{\tt {\sl \%~~Define~the~phantom}}\\
23{\tt~~~}{\tt }\\
24{\tt~~~}{\tt pht\_pos~=~[0~0~40]~/~1000;~~~{\sl \%~~The~position~of~the~phantom}}\\
25{\tt~~~}{\tt }\\
26{\tt~~~}{\tt [m~n]~=~size(pht\_pos);}\\
27{\tt~~~}{\tt pht\_amp~=~20*ones(m,1);~~~~~~{\sl \%~~The~amplitude~of~the~back-scatter}}\\
28{\tt~~~}{\tt }\\
29{\tt~~~}{\tt }\\
30{\tt~~~}{\tt {\sl \%~~Define~the~focus~}}\\
31{\tt~~~}{\tt }\\
32{\tt~~~}{\tt focus\_r~=~[1:max(sqrt(pht\_pos(:,1).\^~2~+~pht\_pos(:,2).\^~2~+~pht\_pos(:,3).\^~2))*1000]'~/~1000;}\\
33{\tt~~~}{\tt T~=~(focus\_r-.5/1000)/c~*2;}\\
34{\tt~~~}{\tt }\\
35{\tt~~~}{\tt }\\
36{\tt~~~}{\tt }\\
37{\tt~~~}{\tt {\sl \%~~Initialize~the~program}}\\
38{\tt~~~}{\tt field\_init(0);}\\
39{\tt~~~}{\tt bft\_init;}\\
40{\tt~~~}{\tt }\\
41{\tt~~~}{\tt {\sl \%~~Set~some~paramters}}\\
42{\tt~~~}{\tt set\_field('c',~c);}\\
43{\tt~~~}{\tt bft\_param('c',~c);}\\
44{\tt~~~}{\tt }\\
45{\tt~~~}{\tt set\_field('fs',~fs);}\\
46{\tt~~~}{\tt bft\_param('fs',~fs);}\\
47{\tt~~~}{\tt }\\
48{\tt~~~}{\tt }\\
49{\tt~~~}{\tt {\sl \%~Create~some~apertures.}}\\
50{\tt~~~}{\tt }\\
51{\tt~~~}{\tt xmt~=~xdc\_linear\_array(no\_elements,width,height,kerf,1,1,[0~0~0]);}\\
52{\tt~~~}{\tt rcv~=~xdc\_linear\_array(no\_elements,width,height,kerf,1,1,[0~0~0]);}\\
53{\tt~~~}{\tt }\\
54{\tt~~~}{\tt xdc~=~bft\_linear\_array(no\_elements,~width,~kerf);}\\
55{\tt~~~}{\tt }\\
56{\tt~~~}{\tt }\\
57{\tt~~~}{\tt {\sl \%~Set~the~impulse~responses}}\\
58{\tt~~~}{\tt xdc\_impulse(rcv,~impulse\_response);}\\
59{\tt~~~}{\tt xdc\_impulse(xmt,~impulse\_response);}\\
60{\tt~~~}{\tt }\\
61{\tt~~~}{\tt xdc\_excitation(xmt,~excitation);}\\
62{\tt~~~}{\tt }\\
63{\tt~~~}{\tt }\\
64{\tt~~~}{\tt {\sl \%~~Define~and~create~the~image}}\\
65{\tt~~~}{\tt sector~=~60~*~pi~/~180;}\\
66{\tt~~~}{\tt no\_lines~=~64;}\\
67{\tt~~~}{\tt d\_theta~=~sector~/~(no\_lines-1);}\\
68{\tt~~~}{\tt theta~=~-(no\_lines-1)~/~2~*~d\_theta;}\\
69{\tt~~~}{\tt }\\
70{\tt~~~}{\tt {\sl \%~~Set~the~delays~for~one~whole~image}}\\
71{\tt~~~}{\tt {\sl \%}}\\
72{\tt~~~}{\tt bft\_no\_lines(no\_lines);}\\
73{\tt~~~}{\tt for~i~=~1~:~no\_lines}\\
74{\tt~~~}{\tt ~~bft\_apodization(xdc,0,hanning(no\_elements)',i);}\\
75{\tt~~~}{\tt {\sl \%~~bft\_sum\_apodization(xdc,0,ones(1,no\_elements),i);}}\\
76{\tt~~~}{\tt ~~focus~=~[sin(theta)*focus\_r,~zeros(length(focus\_r),1),~cos(theta)*focus\_r];}\\
77{\tt~~~}{\tt ~~bft\_center\_focus([0~0~0],i);}\\
78{\tt~~~}{\tt ~~bft\_focus(xdc,~T,~focus,i);}\\
79{\tt~~~}{\tt ~~theta~=~theta~+~d\_theta;}\\
80{\tt~~~}{\tt end}\\
81{\tt~~~}{\tt }\\
82{\tt~~~}{\tt }\\
83{\tt~~~}{\tt {\sl \%}}\\
84{\tt~~~}{\tt {\sl \%~~Allocate~memory~for~the~image}}\\
85{\tt~~~}{\tt {\sl \%}}\\
86{\tt~~~}{\tt Rmax~=~max(sqrt(pht\_pos(:,1).\^~2~+~pht\_pos(:,2).\^~2~~+~pht\_pos(:,3).\^~2))~+~10/1000;}\\
87{\tt~~~}{\tt Rmin~=~min(sqrt(pht\_pos(:,1).\^~2~+~pht\_pos(:,2).\^~2~~+~pht\_pos(:,3).\^~2))~-~10/1000;}\\
88{\tt~~~}{\tt if~(Rmin~<~0)~Rmin~=~0;~end;}\\
89{\tt~~~}{\tt Tmin~=~2*Rmin~/~c;~Tmax~=~2*Rmax~/~c;}\\
90{\tt~~~}{\tt Smin~=~floor(Tmin~*~fs);~Smax~=~ceil(Tmax~*~fs);}\\
91{\tt~~~}{\tt }\\
92{\tt~~~}{\tt no\_rf\_samples~=~Smax~-~Smin~+~1;}\\
93{\tt~~~}{\tt }\\
94{\tt~~~}{\tt bf\_image~=~zeros(no\_rf\_samples,~no\_lines);}\\
95{\tt~~~}{\tt }\\
96{\tt~~~}{\tt {\sl \%}}\\
97{\tt~~~}{\tt {\sl \%~Make~one~low-resolution~image~at~a~time~and~sum~them}}\\
98{\tt~~~}{\tt {\sl \%}}\\
99{\tt~~~}{\tt }\\
100{\tt~~}{\tt xdc\_focus\_times(xmt,0,zeros(1,no\_elements));~}\\
101{\tt~~}{\tt xdc\_focus\_times(rcv,0,zeros(1,no\_elements));}\\
102{\tt~~}{\tt for~emission\_no~=~1:no\_elements}\\
103{\tt~~}{\tt ~~disp(['emission~no:~'~num2str(emission\_no)]);~~}\\
104{\tt~~}{\tt ~~xdc\_apodization(xmt,0,[zeros(1,emission\_no-1)~1~zeros(1,~no\_elements~-~emission\_no)]);}\\
105{\tt~~}{\tt }\\
106{\tt~~}{\tt ~~[scat,~start\_time]~=~calc\_scat\_multi~(xmt,~rcv,~pht\_pos,~pht\_amp);}\\
107{\tt~~}{\tt }\\
108{\tt~~}{\tt ~~start\_sample~=~floor(start\_time~*~fs~+~0.5);}\\
109{\tt~~}{\tt ~~end\_sample~=~start\_sample~+~max(size(scat))-1;}\\
110{\tt~~}{\tt ~~}\\
111{\tt~~}{\tt ~~scat~=~[zeros(start\_sample~-~Smin,~no\_elements);~scat;~zeros(Smax~-~end\_sample,no\_elements)];}\\
112{\tt~~}{\tt ~}\\
113{\tt~~}{\tt ~~start\_time~=~Tmin;}\\
114{\tt~~}{\tt ~~beamformed~=~bft\_beamform(start\_time,scat);}\\
115{\tt~~}{\tt ~~bf\_image~=~bft\_add\_image(bf\_image,~beamformed,~emission\_no,~start\_time);}\\
116{\tt~~}{\tt end}\\
117{\tt~~}{\tt }\\
118{\tt~~}{\tt }\\
119{\tt~~}{\tt }\\
120{\tt~~}{\tt {\sl \%~~Release~the~allocated~memory}}\\
121{\tt~~}{\tt }\\
122{\tt~~}{\tt field\_end}\\
123{\tt~~}{\tt bft\_end}\\
124{\tt~~}{\tt }\\
125{\tt~~}{\tt }\\
126{\tt~~}{\tt }\\
127{\tt~~}{\tt {\sl \%}}\\
128{\tt~~}{\tt {\sl \%~~Dispplay~the~image}}\\
129{\tt~~}{\tt {\sl \%}}\\
130{\tt~~}{\tt }\\
131{\tt~~}{\tt bf\_image~=~abs(hilbert(bf\_image));~~~~~~~~~~~~~~~~~~{\sl \%~Envelope~detection}}\\
132{\tt~~}{\tt bf\_image~=~bf\_image~/~max(max(bf\_image));}\\
133{\tt~~}{\tt }\\
134{\tt~~}{\tt figure;}\\
135{\tt~~}{\tt imagesc([-sector/2~sector/2]*180/pi,[Rmin~Rmax]*1000,20*log10(bf\_image~+~0.001))}\\
136{\tt~~}{\tt axis('image')}\\
137{\tt~~}{\tt xlabel('Angle~[deg]');}\\
138{\tt~~}{\tt ylabel('Axial~distance~[mm]')}\\
139{\tt~~}{\tt title('Beamformed~by~BFT~');}\\
140{\tt~~}{\tt clc}\\
141{\tt~~}{\tt disp(['~'~10~10~10~10~]);}\\
142{\tt~~}{\tt disp([9~'*****************************************************']);}\\
143{\tt~~}{\tt disp([9~'*~~~~~~~~~~~~~~~~~~~~~~~~~~~~~~~~~~~~~~~~~~~~~~~~~~~*']);}\\
144{\tt~~}{\tt disp([9~'*~The~image~beamformed~by~BFT~is~in~"bf\_image"~~~~~~*']);}\\
145{\tt~~}{\tt disp([9~'*~~~~~~~~~~~~~~~~~~~~~~~~~~~~~~~~~~~~~~~~~~~~~~~~~~~*'])}\\
146{\tt~~}{\tt disp([9~'*****************************************************']);}\\
147{\tt~~}{\tt disp(['~'~10~10~]);}\\
148{\tt~~}{\tt }\\
}


\end{document}
