% This is the preamble

\documentclass[a4paper]{article}
\usepackage[margin=1in]{geometry}

\usepackage{fancyhdr}
\pagestyle{fancy}
\lhead{DSP HW 1}
\rhead{\thepage}
\renewcommand{\headrulewidth}{0.4pt}
\renewcommand{\footrulewidth}{0.4pt}

\usepackage{mathtools}
\DeclarePairedDelimiter{\ceil}{\lceil}{\rceil}
\DeclarePairedDelimiter\floor{\lfloor}{\rfloor}

\usepackage[utf8x]{inputenc}
\usepackage{amsmath} % For math formatting
\usepackage{graphicx}
\usepackage{hyperref} 
\usepackage{tcolorbox}
\usepackage{commath}
\usepackage{xcolor}
\hypersetup{
    colorlinks,
    linkcolor={red!50!black},
    citecolor={blue!50!black},
    urlcolor={blue!80!black}
}
\usepackage{multicol}
\usepackage{amssymb}

\newcommand{\op}[2]{#1\{#2\}}

\title{Understanding Convolution}
\author{David Egolf}
\date{September 12, 2016}
% Where the document starts
\begin{document}
\maketitle

\section*{Definition}
The convolution of two sequences $x[n]$ and $h[n]$:
\begin{align*}
(x*h)[n] = \sum_{k=-\infty}^{\infty}x[k]h[n-k]
\end{align*}
Intuitively, we are placing a shifted copy of the sequence $h$ centered at $x = k$, and multiplying this by the $k_{th}$ element in the sequence $x[n]$. We do this for all elements $x[k]$ in the input sequence and add the results.
\\\\
Note that convolution is commutative, distributive, and associative.
\section*{Output of LTI System}
Assume $T$ is a linear time invariant system. Then:
\begin{align*}
\op{T}{\delta[n]} &= h[n] \text{ (impulse response)}\\
\implies \op{T}{x[n]} &= \sum_{k=-\infty}^{k=\infty}x[k]h[n-k]
\end{align*}
\section*{Model Ultrasound System as LTI System}
Consider a single ultrasound transducer, and assume that we use it to transmit a signal, which is then reflected and received by the transducer. Let us define an ultrasound system $U$ that maps from transducer input excitation to the final signal decoded by the transducer:
\begin{align*}
U = T_x \circ M_x \circ R_x
\end{align*}
where $T_x$ is the transmission operator, $M_x$ is the reflection operator (acts like a ``mirror"), and $R_x$  is the receiving operator.
\\\\
If we ignore the transmission delay, and assume that the reflected signal is identical to the transmitted signal up to a change in amplitude, then:
\begin{align*}
\op{M_x}{x[n]} = A \cdot x[n]
\end{align*}
where $A \in \mathbb{R}$.
\\\\
In our simulations we assume that the both $T_x$ and $R_x$ are LTI systems, with the same impulse response. Call this common transducer impulse response $h$. 
\\\\
Using these definitions, we can calculate the output of the ultrasound system:
\begin{align*}
\op{U}{x[n]} &= \op{T_x \circ M_x \circ R_x}{x[n]} \\
&= h[n] *(A \cdot h[n] * x[n]) \\
&= A \cdot \ (h[n] * h[n]) *x[n]
\end{align*}
where we have used the fact that convolution is commutative.
\clearpage
\subsection*{Motivation}
So, in order to understand the action of the ultrasound system, it would be useful to understand the properties of $h[n] * h[n]$, since this is the impulse response of the entire system (up to a scalar multiple).
\section*{Problem Statement}
Investigate the properties of  the self convolution $(h*h)[n]$ of a sequence $h: \mathbb{Z} \rightarrow \mathbb{R}$, in the context of an ultrasound system.
\section*{Solution}
\subsection*{Causal}
I assume there is no noise in the ultrasound system to be modeled. I assume that in a noise free ultrasound system, the system will not begin to transmit data prior to excitation, and the system will not begin to receive data prior to a transmitted signal hitting the receiver. Therefore:
\begin{align*}
h[n] = 0 \text{ for } n < 0 \implies (h*h)[n] = 0 \text{ for } n < 0 
\end{align*}
This implies that the LTI system  $h*h$ is causal. 
\subsection*{Stable}
I assume that if we stop exciting the transducer, then after a finite amount of time the ultrasound receiver will stop receiving anything. That is:
\begin{align*}
(h*h)[n] = 0 \text{ for } n \geq N
\end{align*}
This implies that we are working with a finite impulse response system, and therefore the system is stable.
\subsection*{Not Memory-less}
The output $y[n]$ depends on all values of the input $h[n]$, not just the current value of $n$. Therefore, the system is not memory-less.
\subsection*{Equation for Output}
The output of the system $(h*h)[n]$ is explicitly:
\begin{align*}
(h*h)[n] &= \sum_{k=-\infty}^{\infty}h[k]h[n-k]
\end{align*}
Since the system is causal, we only need to sum over the terms where $k \geq 0$ and $n - k \geq 0 \implies k \leq n$:
\begin{align*}
(h*h)[n] &= \sum_{k=0}^{n}h[k]h[n-k]
\end{align*}
To get some intuition, we write out this sum explicitly in the case when $h[0] = 1, h[1] = 2, h[2] = 3, h[3] = 4, h[4] = 5$ and $h[j] = 0$ for all other $j \in \mathbb{Z}$:
\begin{align*}
(h*h)[n] &= h[0]h[n] + h[1]h[n-1] + h[2]h[n-2] + h[3]h[n-3] + h[4]h[n-4]
\end{align*}
\clearpage
\subsection*{Nonzero Output Region As Function of Length}
Let $L$ be an integer called the ``length" of the impulse response. We provide elements $h[0], h[1], ..., h[L-1]$ to MATLAB when specifying the impulse response. We require $L \geq 1$ and $h[n] = 0$ for all $n \geq L$.
\\\\
We assume that our impulse response starts at zero and ends at zero, so set $h[0] = h[L-1] = 0$.
\\\\
Using this information, we can rewrite the form of the output $(h*h)[n]$. We are interesting in determining exactly at which times the output can be nonzero. The output $(h*h)[n]$ will be zero at $n$ if:
\begin{align*}
h[k]h[n-k] = 0 \text{ for } k = 0,1,..,n
\end{align*}
Since we assume $h[0] = 0$ and $(h*h)[n] = 0$ for all $n \geq L -1$, we can reduce the number of terms under consideration. Specifically, if $n \geq L-1$, then the output is zero, and if $n \leq 0$ then the output is zero. So, it remains to consider the cases in which $1 \leq n \leq L -2$. These are the only cases in which would possibly get nonzero output.
\\\\ 
We can further restrict these cases by realizing that if $L \leq 2$, then the entire sequence is zero and so the output will be zero. So, we only need to consider the cases $1 \leq n \leq L -2$ where $L \geq 3$. We want to know for which of these $n$ values we have a chance for nonzero output, as a function of $L$ (clearly the maximum $n$ for nonzero output will increase with $L$).
\\\\
Our strategy is to start small and search for a pattern:
\\\\
If $n = 1$:
\begin{align*}
h[k]h[n-k] = h[k]h[1-k]
\end{align*}
Since $1 - k \leq 0$ for $k = 1,..,L-2$, the output is always zero in this case. As a result, we only need to consider the possible nonzero cases as consisting of $2 \leq n \leq L -2$.
\\\\
If $n = 2$:
\begin{align*}
h[k]h[n-k] = h[k]h[2-k] \\
\end{align*}
The possible nonzero $h[i]$ range is $i = 1,2..,L-2$. Checking when we are in this range:
\begin{align*}
1 \leq 2-k \leq L -2 \\
\implies k &\leq 1 \text{ and}\\
\implies 4 &\leq L + k \implies k \geq 4 - L \\
\text{Together: } \\
4 - L \leq k \leq 1
\end{align*}
In order to satisfy this inequality, we need:
\begin{align*}
4 - L \leq 1 \implies L \geq 3
\end{align*}
We also need:
\begin{align*}
4 - L  \leq k_{max}, 1  \geq k_{min}
\end{align*}
Where $k_{max}$ is the largest value $k$ can take on, and $k_{min}$ is the smallest value $k$ can take on, while preserving the fact that $h[k]$ might be nonzero. From our work before, $k_{max} = L - 2$ and $k_{min} = 1$.
\\\\
So, we need:
\begin{align*}
4 - L \leq L -2, 1 \geq 1 \\
\implies 2L \geq 6 \implies L \geq 3
\end{align*}
\\\\
So, $(h*h)[2]$ has a chance to be nonzero when $L \geq 3$.
\clearpage
Let's generalize this argument, setting $n = a$:
\begin{align*}
h[k]h[n-k] = h[k]h[a-k] \\
\end{align*}
The possible nonzero $h[i]$ range is $i = 1,2..,L-2$. Checking when we are in this range:
\begin{align*}
1 \leq a-k \leq L -2 \\
&\implies k \leq a - 1 \text{ and}\\
&\implies a + 2 \leq L + k \implies k \geq a + 2 - L \\
\text{Together: } \\
a + 2 - L \leq k \leq a - 1
&\implies L \geq 3
\end{align*}
We also need:
\begin{align*}
a + 2 - L  \leq k_{max}, a - 1  \geq k_{min}
\end{align*}
Where $k_{max}$ is the largest value $k$ can take on, and $k_{min}$ is the smallest value $k$ can take on, while preserving the chance that $h[k]$ might be nonzero. From our work before, $k_{max} = L - 2$ and $k_{min} = 1$.
\\\\
So, we need:
\begin{align*}
a +2 - L \leq L -2,~a-1 \geq 1 \\
\implies 2L \geq a + 4 \implies L \geq \frac{a+4}{2}
\end{align*}
\\\\
Substituting $n = a$, we find $(h*h)[n]$ has a chance to be nonzero when $L \geq \frac{n+4}{2}$. That means when we are computing the sum to get the response, we only need to add terms from $n = 1$ to $2L - 4$.
\end{document}
